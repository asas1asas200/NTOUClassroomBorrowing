%!TEX program=xelatex
\documentclass{article}

% Chinese font
\usepackage{xeCJK}
\usepackage{hyperref}

\setCJKmainfont{[kaiu.ttf]}
\setmainfont{[times new roman.ttf]}

\usepackage{colortbl}
\usepackage{xcolor}

\definecolor{LightGray}{gray}{0.8}
\newcolumntype{a}{>{\columncolor{LightGray}}c}
\newcolumntype{C}{>{\centering\arraybackslash}X}
\newcolumntype{M}[1]{>{\centering\arraybackslash}m{#1}}
\newcolumntype{A}[1]{>{\columncolor{LightGray}}m{#1}}

\usepackage{tabularx}
\renewcommand\tabularxcolumn[1]{m{#1}}
\usepackage{makecell}

\renewcommand*\contentsname{目錄}

\usepackage{tocloft}
\usepackage{indentfirst}

\begin{document}
\begin{titlepage}
	\centering

	{\huge 海大教室借用平台}

	\vfill

	{\huge 測試文件}

	\vfill

	\begin{Large}
		\begin{center}
			\begin{tabular}{| a | c |}
				\hline
				專案名稱 & 海大教室借用平台               \\ \hline
				撰寫日期 & \today                 \\ \hline
				發展者  & \makecell{曾昱翔、林暐傑、陳鈺翔、 \\張銀軒、黃見弘} \\ \hline
			\end{tabular}
		\end{center}
	\end{Large}
\end{titlepage}


\addcontentsline{toc}{section}{版次變更紀錄}
\section*{版次變更紀錄}

\begin{tabularx}{\textwidth}{| c | X | X |}
	\rowcolor{LightGray}
	\hline
	版次  & 變更項目   & 變更日期       \\ \hline
	0.1 & 初版     & 2022/10/04 \\ \hline
	0.2 & 文件補完   & 2023/01/02 \\ \hline
	0.3 & 更新測試結果 & 2023/01/10 \\ \hline
	    &        &            \\ \hline
	    &        &            \\ \hline
	    &        &            \\ \hline
\end{tabularx}

\newpage

\begin{center}
	\tableofcontents
\end{center}

\newpage

\section[測試目的與接受準則(OBJECTIVES AND ACCEPTANCE CRITERIA)]{測試目的與接受準則(Objectives and Acceptance Criteria)}

\subsection[系統範圍(SYSTEM SCOPE)]{系統範圍(System Scope)}

本測試文件基於 \href{https://github.com/asas1asas200/NTOUClassroomBorrowing/releases/tag/v0.3}{海大教室借用系統 v0.3} 版本進行測試,測試範圍包含所有功能模組。

達成需求規格中部份功能的測試目的,如下:

\begin{itemize}
	\item 功能需求:
	      \begin{itemize}
		      \item 帳號註冊
		      \item 帳號登入
		      \item 個人資料管理
		      \item 審核個人資料
		      \item 教室狀態
		      \item 教室時段篩選
		      \item 教室借用歷程
		      \item 教室立即借用
		      \item 教室借用進度查詢
		      \item 預約審核
		      \item 鑰匙歸還登記
		      \item 教室編輯
	      \end{itemize}
	\item 非功能需求:
	      \begin{itemize}
		      \item 延遲時間不超過一分鐘
		      \item 密碼以雜湊方式儲存、傳輸
		      \item 查詢以前端實作
		      \item 系統人數上線至少 100 人
		      \item 最多 1000ms 的回應時間
	      \end{itemize}
\end{itemize}

\subsection[測試接受準則(TEST ACCEPTANCE CRITERIA)]{測試接受準則(Test Acceptance Criteria)}

本測試計畫需要滿足下面的測試接受準則:

\begin{enumerate}
	\color{blue}
	\item 測試程序需要依照本測試計劃所訂定的程序進行,所有測試結果需要能符合預期測試結果方能接受。
	      \begin{itemize}
		      \item 當測試案例未通過時,相關模組開發之負責人需要進行程式修改(修復bug或改動功能),以能讓此案例重新通過測試。
		      \item 重新進行測試時,測試人員需確認其他可能受影響的案例仍可正確執行。
	      \end{itemize}
\end{enumerate}


\newpage

\section[測試環境(TESTING ENVIRONMENT)]{測試環境(Testing Environment)}

\subsection[硬體需求(HARDWARE SPECIFICATION AND CONFIGURATION)]{硬體需求(Hardware Specification and Configuration)}

Server 端硬體需求如下:

\bigskip

\begin{tabularx}{\textwidth}{| c | C | c | C | M{0.15\textwidth} |}
	\rowcolor{LightGray}
	\hline
	項次 & 名稱             & 數量 & 規格            & 備註 \\ \hline
	1  & 一般桌上型電腦或機架式伺服器 & 1  & 限制 x86\_64 平台 &    \\ \hline
	   &                &    &               &    \\ \hline
	   &                &    &               &    \\ \hline
\end{tabularx}

\bigskip

Client 端硬體需求如下:

\bigskip

\begin{tabularx}{\textwidth}{| c | C | c | C | M{0.15\textwidth} |}
	\rowcolor{LightGray}
	\hline
	項次 & 名稱            & 數量 & 規格    & 備註 \\ \hline
	1  & 一般桌上型電腦或智慧型手機 & 1  & 具上網功能 &    \\ \hline
	   &               &    &       &    \\ \hline
	   &               &    &       &    \\ \hline
\end{tabularx}

\subsection[軟體需求(SOFTWARE SPECIFICATION AND CONFIGURATION)]{軟體需求(Software Specification and Configuration)}

Server 端軟體需求如下:

\bigskip

\begin{tabularx}{\textwidth}{| c | C | c | C | M{0.15\textwidth} |}
	\rowcolor{LightGray}
	\hline
	項次 & 名稱      & 數量 & 規格                        & 備註                              \\ \hline
	1  & 作業系統    & 1  & Linux 64-bit              & 測試環境為 Arch Linux 和 Ubuntu 22.04 \\ \hline
	2  & Docker  & 1  & Docker 20.10.22           &                                 \\ \hline
	3  & Node.js & 1  & v19.3.0、v14.x、v16.x、v18.x &                                 \\ \hline
\end{tabularx}

\bigskip

Client 端軟體需求如下:

\bigskip

\begin{tabularx}{\textwidth}{| c | C | c | C | M{0.15\textwidth} |}
	\rowcolor{LightGray}
	\hline
	項次 & 名稱  & 數量 & 規格                                 & 備註 \\ \hline
	1  & 瀏覽器 & 1  & 具解析 HTML5 + CSS3 和執行 JavaScript 功能 &    \\ \hline
	   &     &    &                                    &    \\ \hline
	   &     &    &                                    &    \\ \hline
\end{tabularx}

\subsection[測試資料來源(TEST DATA SOURCE)]{測試資料來源(Test Data Source)}

測試資料由測試人員根據測試案例的需求自行產生。

\subsection[測試工具與設備(TOOLS AND EQUIPMENT)]{測試工具與設備(Tools and Equipment)}

\begin{itemize}
	\item 單元測試:以 Jest 和 Supertest 為主要工具,並且搭配 Github Action 作為 CI/CD 工具,進而保證每次的更改不影響先前的結果。
	\item 壓力測試:以 JMeter 進行隨機的壓力測試,保證系統在一定流量時不會崩潰。
\end{itemize}

\newpage

\section[測試案例(TEST CASES)]{測試案例(Test Cases)}

\begin{tabularx}{\textwidth}{| A{0.3\textwidth} | X |}
	\hline
	Identification  & NCB-TC-001       \\ \hline
	Name            & 帳號註冊             \\ \hline
	Reference       & 帳號註冊             \\ \hline
	Severity        & 高                \\ \hline
	Instructions    & 進入帳號註冊頁面輸入資料註冊帳號 \\ \hline
	Expected result & 成功註冊一個新帳號        \\ \hline
	Cleanup         & 無                \\ \hline
\end{tabularx}

\bigskip

\begin{tabularx}{\textwidth}{| A{0.3\textwidth} | X |}
	\hline
	Identification  & NCB-TC-002    \\ \hline
	Name            & 帳號登入          \\ \hline
	Reference       & 帳號登入          \\ \hline
	Severity        & 高             \\ \hline
	Instructions    & 在登入頁面登入已註冊的帳號 \\ \hline
	Expected result & 成功以指定帳號登入     \\ \hline
	Cleanup         & 按登出按鈕以登出      \\ \hline
\end{tabularx}

\bigskip

\begin{tabularx}{\textwidth}{| A{0.3\textwidth} | X |}
	\hline
	Identification  & NCB-TC-003       \\ \hline
	Name            & 個人資料管理           \\ \hline
	Reference       & 個人資料管理           \\ \hline
	Severity        & 中                \\ \hline
	Instructions    & 進入個人資料編輯頁面,編輯後儲存 \\ \hline
	Expected result & 成功編輯個人資料         \\ \hline
	Cleanup         & 無                \\ \hline
\end{tabularx}

\bigskip

\begin{tabularx}{\textwidth}{| A{0.3\textwidth} | X |}
	\hline
	Identification  & NCB-TC-004         \\ \hline
	Name            & 審核個人資料             \\ \hline
	Reference       & 審核個人資料             \\ \hline
	Severity        & 高                  \\ \hline
	Instructions    & 管理員在管理介面審核使用者之個人資料 \\ \hline
	Expected result & 成功改變指定帳戶的審核狀態      \\ \hline
	Cleanup         & 無                  \\ \hline
\end{tabularx}

\bigskip

\begin{tabularx}{\textwidth}{| A{0.3\textwidth} | X |}
	\hline
	Identification  & NCB-TC-005                     \\ \hline
	Name            & 查看教室狀態                         \\ \hline
	Reference       & 教室狀態、教室時段篩選、教室借用歷程             \\ \hline
	Severity        & 中                              \\ \hline
	Instructions    & 在主頁面可以看到教室當前的狀態,包含鑰匙、指定日期之借用狀態 \\ \hline
	Expected result & 可以看到教室當前鑰匙狀態和指定日期之借用情況等資訊      \\ \hline
	Cleanup         & 無                              \\ \hline
\end{tabularx}

\bigskip

\begin{tabularx}{\textwidth}{| A{0.3\textwidth} | X |}
	\hline
	Identification  & NCB-TC-006                 \\ \hline
	Name            & 教室立即借用                     \\ \hline
	Reference       & 教室立即借用                     \\ \hline
	Severity        & 高                          \\ \hline
	Instructions    & 選擇日期和教室後申請借用教室,並按照表格填入應填資訊 \\ \hline
	Expected result & 成功送出教室借用申請                 \\ \hline
	Cleanup         & 無                          \\ \hline
\end{tabularx}

\bigskip

\begin{tabularx}{\textwidth}{| A{0.3\textwidth} | X |}
	\hline
	Identification  & NCB-TC-007                     \\ \hline
	Name            & 教室借用進度查詢                       \\ \hline
	Reference       & 教室借用進度查詢                       \\ \hline
	Severity        & 中                              \\ \hline
	Instructions    & 查詢教室借用申請的借用進度                  \\ \hline
	Expected result & 網頁上會顯示當前借用進度,申請中、核可、拒絕、借用中或已完成 \\ \hline
	Cleanup         & 無                              \\ \hline
\end{tabularx}

\bigskip

\begin{tabularx}{\textwidth}{| A{0.3\textwidth} | X |}
	\hline
	Identification  & NCB-TC-008     \\ \hline
	Name            & 教室借用審核         \\ \hline
	Reference       & 預約審核           \\ \hline
	Severity        & 高              \\ \hline
	Instructions    & 管理員在管理介面審核借用申請 \\ \hline
	Expected result & 成功同意/拒絕指定的借用申請 \\ \hline
	Cleanup         & 無              \\ \hline
\end{tabularx}

\bigskip

\begin{tabularx}{\textwidth}{| A{0.3\textwidth} | X |}
	\hline
	Identification  & NCB-TC-009            \\ \hline
	Name            & 鑰匙狀態管理                \\ \hline
	Reference       & 鑰匙歸還登記                \\ \hline
	Severity        & 高                     \\ \hline
	Instructions    & 管理員在管理介面操作鑰匙狀態        \\ \hline
	Expected result & 指定教室的鑰匙狀態會被更新,可從主畫面確認 \\ \hline
	Cleanup         & 無                     \\ \hline
\end{tabularx}

\bigskip

\begin{tabularx}{\textwidth}{| A{0.3\textwidth} | X |}
	\hline
	Identification  & NCB-TC-010            \\ \hline
	Name            & 大樓/樓層/教室編輯            \\ \hline
	Reference       & 教室編輯                  \\ \hline
	Severity        & 低                     \\ \hline
	Instructions    & 管理員在管理介面編輯大樓/樓層/教室的資訊 \\ \hline
	Expected result & 編輯結果成功存入資料庫           \\ \hline
	Cleanup         & 無                     \\ \hline
\end{tabularx}

\bigskip

\begin{tabularx}{\textwidth}{| A{0.3\textwidth} | X |}
	\hline
	Identification  & NCB-TC-011      \\ \hline
	Name            & 系統狀態更新延遲        \\ \hline
	Reference       & 延遲時間不超過一分鐘      \\ \hline
	Severity        & 低               \\ \hline
	Instructions    & 更新鑰匙狀態後手動查看是否更新 \\ \hline
	Expected result & 一分鐘內主頁同步更新狀態    \\ \hline
	Cleanup         & 無               \\ \hline
\end{tabularx}

\bigskip

\begin{tabularx}{\textwidth}{| A{0.3\textwidth} | X |}
	\hline
	Identification  & NCB-TC-012            \\ \hline
	Name            & 密碼安全                  \\ \hline
	Reference       & 密碼以雜湊方式儲存、傳輸          \\ \hline
	Severity        & 中                     \\ \hline
	Instructions    & 於資料庫和封包紀錄中查看密碼如何存放及傳輸 \\ \hline
	Expected result & 皆無明文形式之密碼             \\ \hline
	Cleanup         & 無                     \\ \hline
\end{tabularx}

\bigskip

\begin{tabularx}{\textwidth}{| A{0.3\textwidth} | X |}
	\hline
	Identification  & NCB-TC-013    \\ \hline
	Name            & 查詢功能以前端實作     \\ \hline
	Reference       & 查詢以前端實作       \\ \hline
	Severity        & 低             \\ \hline
	Instructions    & 在網頁操作查詢功能     \\ \hline
	Expected result & 查詢時瀏覽器不會有換頁行為 \\ \hline
	Cleanup         & 無             \\ \hline
\end{tabularx}

\bigskip

\begin{tabularx}{\textwidth}{| A{0.3\textwidth} | X |}
	\hline
	Identification  & NCB-TC-014                     \\ \hline
	Name            & 壓力測試                           \\ \hline
	Reference       & 系統人數上線至少 100 人、最多 1000ms 的回應時間 \\ \hline
	Severity        & 低                              \\ \hline
	Instructions    & 以壓力測試軟體進行測試及分析                 \\ \hline
	Expected result & 最多回應時間不超過 1000ms               \\ \hline
	Cleanup         & 無                              \\ \hline
\end{tabularx}

\newpage

\section[測試工作指派與時程(PERSONNEL AND SCHEDULE)]{測試工作指派與時程(Personnel and Schedule)}

\subsection[測試成員(PERSONNEL)]{測試成員(Personnel)}

本專案分工上參考扁平化組織的實作,並遵循開源文化的精神,由成員們自由認領工作。

\newpage

\section[測試結果與分析(TEST RESULTS AND ANALYSIS)]{測試結果與分析(Test Results and Analysis)}

\subsection[測試結果(TEST RESULTS)]{測試結果(Test Results)}

\begin{tabularx}{\textwidth}{| m{0.3\textwidth} | m{0.2\textwidth} | X |}
	\rowcolor{LightGray}
	\hline
	測試案例編號     & \makecell{測試結果    \\ (Pass/Fail)} & 註解 \\ \hline
	NCB-TC-001 & Pass            & \\ \hline
	NCB-TC-002 & Pass            & \\ \hline
	NCB-TC-003 & \color{red}Fail & \\ \hline
	NCB-TC-004 & Pass            & \\ \hline
	NCB-TC-005 & Pass            & \\ \hline
	NCB-TC-006 & Pass            & \\ \hline
	NCB-TC-007 & Pass            & \\ \hline
	NCB-TC-008 & Pass            & \\ \hline
	NCB-TC-009 & Pass            & \\ \hline
	NCB-TC-010 & Pass            & \\ \hline
	NCB-TC-011 & Pass            & \\ \hline
	NCB-TC-012 & \color{red}Fail & \\ \hline
	NCB-TC-013 & Pass            & \\ \hline
	NCB-TC-014 & Pass            & \\ \hline
	\rowcolor{LightGray}
	RATE       & 85.7\%          & \\ \hline
\end{tabularx}

\subsection[缺失報告(DEFECT TRACKING)]{缺失報告(Defect Tracking)}

\begin{tabularx}{\textwidth}{| c | c | C | c | c | c | c |}
	\rowcolor{LightGray}
	\hline
	缺失  & 缺失     & 缺失                         & 測試案例       & 缺失  & 修復      & 修復                                                                            \\
	\rowcolor{LightGray}
	編號  & 嚴重性    & 說明                         & 編號         & 負責人 & 狀態      & 說明                                                                            \\ \hline
	001 & Medium & 變更密碼時需輸入舊密碼,但程式不會檢查舊密碼是否正確 & NCB-TC-003 & 張銀軒 & Closed  & \href{https://github.com/asas1asas200/NTOUClassroomBorrowing/issues/62}{\#62} \\ \hline
	002 & High   & 密碼以明文形式傳輸                  & NCB-TC-012 & 林暐傑 & Ongoing & \href{https://github.com/asas1asas200/NTOUClassroomBorrowing/pull/45}{\#45}   \\ \hline
	    &        &                            &            &     &         &                                                                               \\ \hline
\end{tabularx}

\newpage

\section[追溯表(TRACEABILITY MATRIX)]{追溯表(Traceability Matrix)}

\begin{tabularx}{\textwidth}{| X | X | X |}
	\hline
	Req. No.       & Test Case \# & Verification \\ \hline
	CBP-AM-LOG-01  & NCB-TC-002   & Verified     \\ \hline
	CBP-AM-REG-01  & NCB-TC-001   & Verified     \\ \hline
	CBP-AM-PIM-01  & NCB-TC-003   & Verified     \\ \hline
	CBP-AM-PIV-01  & NCB-TC-004   & Verified     \\ \hline
	CBP-CBI-LOG-01 & NCB-TC-005   & Verified     \\ \hline
	CBP-CBI-LOG-03 & NCB-TC-005   & Verified     \\ \hline
	CBP-CBI-LOG-04 & NCB-TC-005   & Verified     \\ \hline
	CBP-CBI-BOR-01 & NCB-TC-006   & Verified     \\ \hline
	CBP-CBI-BOR-03 & NCB-TC-007   & Verified     \\ \hline
	CBP-CBI-BOR-3  & NCB-TC-008   & Verified     \\ \hline
	CBP-CBI-KR-1   & NCB-TC-009   & Verified     \\ \hline
	CBP-CBI-CR-1   & NCB-TC-010   & Verified     \\ \hline
	CBP-NFR-01     & NCB-TC-011   & Verified     \\ \hline
	CBP-NFR-02     & NCB-TC-012   & Unverified   \\ \hline
	CBP-NFR-03     & NCB-TC-013   & Verified     \\ \hline
	CBP-NFR-05     & NCB-TC-014   & Verified     \\ \hline
	CBP-NFR-06     & NCB-TC-014   & Verified     \\ \hline
\end{tabularx}

\end{document}