%!TEX program=xelatex
\documentclass{article}

% Chinese font
\usepackage{xeCJK}
\usepackage{hyperref}

\setCJKmainfont{[kaiu.ttf]}
\setmainfont{[times new roman.ttf]}

\usepackage{colortbl}
\usepackage{xcolor}

\definecolor{LightGray}{gray}{0.8}
\newcolumntype{a}{>{\columncolor{LightGray}}c}

\usepackage{tabularx}
\usepackage{makecell}

\renewcommand*\contentsname{目錄}

\usepackage{graphicx}
\graphicspath{{./img/SDD}}

\usepackage{indentfirst}

\begin{document}
\begin{titlepage}
	\centering

	{\huge 海大教室借用平台}

	\vfill

	{\huge 設計文件}

	\vfill

	\begin{Large}
		\begin{center}
			\begin{tabular}{| a | c |}
				\hline
				專案名稱 & 海大教室借用平台               \\ \hline
				撰寫日期 & \today                 \\ \hline
				發展者  & \makecell{曾昱翔、林暐傑、陳鈺翔、 \\張銀軒、黃見弘} \\ \hline
			\end{tabular}
		\end{center}
	\end{Large}
\end{titlepage}


\addcontentsline{toc}{section}{版次變更紀錄}
\section*{版次變更紀錄}

\begin{tabularx}{\textwidth}{| c | X | X |}
	\rowcolor{LightGray}
	\hline
	版次    & 變更項目      & 變更日期       \\ \hline
	0.1   & 初版        & 2022/10/04 \\ \hline
	0.1.1 & 新增系統模型與架構 & 2022/11/28 \\ \hline
	      &           &            \\ \hline
	      &           &            \\ \hline
	      &           &            \\ \hline
	      &           &            \\ \hline
\end{tabularx}

\newpage

\begin{center}
	\tableofcontents
\end{center}

\newpage

\section[系統模型與架構(SYSTEM MODEL/SYSTEM ARCHITECTURE)]{系統模型與架構(System Model/System Architecture)}

\centerline{\includegraphics[width=\textwidth]{HighLevelArchitecture.png}}

\bigskip

前後端透過瀏覽器的 HTTP request 和 axios 提供的 request 方法與後端溝通,後端透過 express 提供的路由來處理前端的要求。

實際傳輸的方式包含了 RESTful API 和直接用 ejs render 後的結果當成網頁內容。

每個頁面透過瀏覽器發送的 request 來得到後端 ejs render 完的 HTML 內容,頁面中需要純前端處理的部分則以 axios 來傳送 request ,接收後端 render 完的 response 後直接操作 DOM 來改變頁面內容。

\subsection{後端架構}

後端架構主要分為兩個部分:後端伺服器和資料庫。

\subsubsection{後端伺服器}

伺服器為 Node.js express 搭配 ejs 來實作,主要負責處理前端的 request 並回傳給前端。

後端架構遵循 express-generator 產生出的 MVC 架構,將後端大致分為三個部分: routes、views 和 models 。

\subsubsection{資料庫}

透過 docker 來管理本地端的資料庫,使用的是 mongodb ,並在 express app 中以 mongoose ODM 來操作資料庫。

資料庫連線分為 \verb|/test| 和 \verb|/NTOUClassroomBorrowing| ,分別用來測試和正式使用。

\subsection{前端架構}

前端透過 Bootstrap 5 和基本的 HTML、CSS、JavaScript 來實作,主要負責處理使用者的操作並將結果傳送給後端。

部份動畫透過 animate.css 提供,純前端處理 request 的部份則以 axios 與後端溝通,收取 response 來更新頁面內容,或是彈出提示訊息。

\newpage

\section[介面需求與設計(INTERFACE REQUIREMENTS AND DESIGN)]{介面需求與設計(Interface Requirements and Design)}

\newpage

\section[流程設計(PROCESS DESIGN)]{流程設計(Process Design)}

\newpage

\section[使用者畫面設計(USER INTERFACE DESIGN)]{使用者畫面設計(User Interface Design)}

\newpage

\section[資料設計(DATA DESIGN)]{資料設計(Data Design)}

\newpage

\section[類別圖設計(CLASS DIAGRAM DESIGN)]{類別圖設計(Class Diagram Design)}

\newpage

\section[實作技術(IMPLEMENTATION LANGUAGE AND PLATFORM)]{實作技術(Implementation Language and Platform)}

\newpage

\section[設計議題(DESIGN ISSUE)]{設計議題(Design Issue)}

\end{document}